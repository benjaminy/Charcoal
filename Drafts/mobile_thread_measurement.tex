

\begin{abstract}

Multithreading is now used in most modern mainstream software \[citation
needed\].  In this paper we report on measurements and analyses of the
threading behavior of dozens of popular applications running on six
popular mobile and desktop platforms.  Relative to previous measurement
studies of threading behavior, we make two primary contributions: We
perform measurements on three popular mobile platforms; previous studies
were conducted only on desktops and servers.  We measure and analyze the
use of threads as event handlers.

Much of the research literature on multithreading focuses exclusively on
the use of threads to keep multi-processor systems busy.  That is, the
extent to which applications exploit thread level parallelism (TLP).
Several previous studies have shown that most popular applications have
precious little TLP.  We quantify the other use of threads: asynchronous
event handling.

blah lots of threads just sit there

Our headline results: Mobile apps have no more TLP than their desktop
cousins.  The average in our study was 1.X.

Most threads are event handler threads.  9X\% of the time when a thread
starts running, it blocks again within Y microseconds

Might have some implications for schedulers.


Multithreaded programming has gone mainstream.  There are two primary
reasons for this shift, which correspond to the two distinct uses of
threads: First, threads are used to take advantage of the processor
parallelism available on multi-core processors.  Multi-cores are found
in all commodity computing platforms, with core counts predicted to rise
into the dozens in low-cost devices in the near future.  Second, threads
are used to allow applications to remain responsive while performing (or
waiting to perform) I/O operations.  Modern mainstream applications have
more of this kind of interactivity than their peers a decade ago because
of the increasing use of networked services and increasing richness of
human interface modalities, especially on mobile platforms.

In this paper we report on how modern applications--both desktop and
mobile--are using threads.  We measured both parallelism and the use of
threads as I/O event handlers.  We find that few applications use more
than a very modest amount of thread level parallelism, which confirms
and extends previous studies.  On the other hand, we find that most
applications use threads as event handlers.  In some cases, applications
have dozens of threads waiting on I/O events simulatenously.

The bulk of this paper is a detailed report on the measurements that we
took regarding multithreading behavior.  We also offer some suggestions
about how systems could be adapted to better support the way
applications use threads.

\end{abstract}
